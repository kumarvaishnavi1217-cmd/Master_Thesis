\chapter{Environment Analysis and Preparation (WP1)}
\label{chap:wp1}

\section{Environment Overview}
\label{ch:wp1_environment}

The environment is implemented as a custom Gym environment and simulates the longitudinal dynamics of a train operating a predefined railway track. The simulation incorporates physical effects such as traction and braking forces, rolling and aerodynamic resistance, track gradients, speed limits, and energy consumption.

Each episode corresponds to a single train journey between two stations. At each decision step, the agent selects a discrete control action, which is internally translated into a continuous traction or braking request. The environment advances the physical simulation and returns an observation, a scalar reward, a termination signal, and auxiliary diagnostic information.

This chapter documents the analysis, verification, and targeted improvement of the
reinforcement learning environment used for autonomous train driving.
The focus of this work package is not on controller design, but on ensuring that the
environment dynamics, state representation, and reward formulation are well-understood,
measurable, and suitable for subsequent learning-based optimization.

\section{Initial Environment Inspection}

The initial step was to execute and inspect the existing reinforcement learning environment to analyze its correct operation and understand its internal structure.

The inspection focused on the following aspects:
\begin{itemize}
    \item Definition of State (observation) and dimensionality
    \item Action space semantics
    \item Episode termination conditions
    \item Reward signal structure
\end{itemize}The inspection focused on the following aspects:


\section{State Space Inspection and Verification}

The observation returned by the environment consists of a structured state composed of
two components:
\begin{enumerate}
    \item A continuous state vector of length 10 representing the current train and journey state.
    \item A fixed-size matrix encoding future speed limit segments.
\end{enumerate}

In total, the environment exposes a 64-dimensional state representation.
To verify correctness and consistency, a dedicated inspection script was implemented to:
\begin{itemize}
    \item Validate state dimensionality
    \item Assign semantic names to each state component
    \item Confirm the numerical normalization ranges
\end{itemize}

This verification confirmed that the observed space implemented matches the documented
design and remains consistent across episode resets and environment steps.

\section{Reward Function Decomposition}

Originally, the environment exposed only a scalar reward value to the learning agent.
To improve interoperability and enable systematic debugging, the reward function was
refactored to expose its individual components via the environment \texttt{info} dictionary.


The total reward consists of multiple terms reflecting different operational objectives:
\begin{itemize}
    \item Safety (e.g., speed limit compliance)
    \item Passenger comfort
    \item Energy efficiency
    \item Punctuality
    \item Accuracies of parking at destination
\end{itemize}

Each component is now explicitly recorded and returned via the environment’s information dictionary at every step. This modification preserves full compatibility with standard reinforcement learning algorithms while enabling detailed post-hoc analysis of agent decisions and trade-offs between competing objectives.

\section{Termination Analysis}

The environment defines several terminal conditions, including successful arrival at the destination, premature stopping before reaching the target, and overshooting the destination. To better understand agent behavior, termination reasons were explicitly logged and analyzed across multiple episodes.

Initial evaluations showed that a significant portion of the episodes ended without a successful arrival at the destination. This analysis revealed that the existing reward structure encouraged prolonged survival rather than task completion, which motivated further reward shaping in later stages of the thesis.

\section{Dataset Generation for Offline Analysis}

To support systematic evaluation and future explainability analyzes, rollout data from training agents are collected continuously and stored in structured datasets. These datasets include state features, actions, rewards, and termination metadata for each step.

The collected data enables inspection of agent behavior, identification of failure modes, and quantitative comparison between different environment configurations and learning strategies.

\section{Summary}

This chapter presented the analysis and preparation of the reinforcement learning environment used for autonomous train driving. The state space was validated, the reward function was decomposed for transparency, termination behavior was analyzed, and datasets were generated for offline evaluation.

These steps establish a reliable and interpretable foundation for the subsequent development of machine learning–based decision models and explainability methods in the following chapters.

